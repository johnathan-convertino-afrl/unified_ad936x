\begin{titlepage}
  \begin{center}

  {\Huge FMCOMMS2-3}

  \vspace{25mm}

  \includegraphics[width=0.90\textwidth,height=\textheight,keepaspectratio]{img/AFRL.png}

  \vspace{25mm}

  \today

  \vspace{15mm}

  {\Large Jay Convertino}

  \end{center}
\end{titlepage}

\tableofcontents

\newpage

\section{Usage}

\subsection{Introduction}

\par
The fmcomms2-3 project builds a FPGA base system for the fmcomms2 and fmcomms3 Analog Devices development boards.
Project targets are listed in \ref{targets}. The base IP for the Analog Devices parts are from the Analog Devices HDL repo.
They have been converted into fusesoc cores and some modifications have been made. Modifications include making the ADC/DAC
routes both use AXIS out of the DMAs. The Intel FPGA targets now uses ad\_data/ad\_clock cores, clock select, and DC filter
to reach functionally on par with Xilinx targets.

\subsection{Dependencies}

\par
The following are the dependencies of the cores.

\begin{itemize}
  \item fusesoc 2.X
  \item iverilog (simulation)
  \item cocotb (simulation)
\end{itemize}

\input{src/fusesoc/depend_fusesoc_info.tex}

\section{Architecture}
\par
The project contains four wrappers

\begin{itemize}
  \item \textbf{system\_wrapper} Contains the top level project module and contains system\_pl\_wrapper and system\_ps\_wrapper.
  \item \textbf{system\_pl\_wrapper} Contains the AD9361 wrapper and any support IP's in the program logic.
  \item \textbf{ad9361\_pl\_wrapper} Contains all program logic IP's dealing with the AD9361.
  \item \textbf{system\_ps\_wrapper} Contains the processor system IP wrappers.
\end{itemize}

\par

Please see \ref{Module Documentation} for more information per target.

\section{Building}

\par
The all fmcomms2-3 core is written in Verilog 2001. They should synthesize in any modern FPGA software. The core comes as a fusesoc packaged core and can be
included in any other core. Be sure to make sure you have meet the dependencies listed in the previous section.

\subsection{fusesoc}
\par
Fusesoc is a system for building FPGA software without relying on the internal project management of the tool. Avoiding vendor lock in to Vivado or Quartus.
These cores, when included in a project, can be easily integrated and targets created based upon the end developer needs. The core by itself is not a part of
a system and should be integrated into a fusesoc based system. Simulations are setup to use fusesoc and are a part of its targets.

\subsection{Source Files}

\input{src/fusesoc/files_fusesoc_info.tex}

\subsection{Targets} \label{targets}

\input{src/fusesoc/targets_fusesoc_info.tex}

\subsection{Directory Guide}

\par
Below highlights important folders from the root of the directory.

\begin{enumerate}
  \item \textbf{docs} Contains all documentation related to this project.
    \begin{itemize}
      \item \textbf{manual} Contains user manual and github page that are generated from the latex sources.
    \end{itemize}
  \item \textbf{a10soc} Contains source files for Arria 10 soc
  \item \textbf{common} Contains source file wrapper for ad9361 core
  \item \textbf{hanpilot} Contains source files for Arria based hanpilot
  \item \textbf{zc702} Contains source files for Xilinx zc702
  \item \textbf{zc706} Contains source files for Xilinx zc706
  \item \textbf{zcu102} Contains source files for Xilinx zcu102
  \item \textbf{zed} Contains source files for Digilent Zedboard
\end{enumerate}

\newpage

\section{Simulation}
\par
There is no simulation at the moment. This is dues to the AD9361 and ARM subsystems. Maybe a future addition with Vexriscv?

\newpage

\section{Module Documentation} \label{Module Documentation}

\par
There project has multiple modules. The targets are the top system wrappers.

\begin{itemize}
\item \textbf{ad9361 system pl}
\item \textbf{hanpilot system pl}
\item \textbf{hanpilot system}
\item \textbf{zc702 system pl}
\item \textbf{zc702 system}
\item \textbf{zc706 system pl}
\item \textbf{zc706 system}
\item \textbf{zcu102 system pl}
\item \textbf{zcu102 system}
\item \textbf{zed system pl}
\item \textbf{zed system}
\item \textbf{a10soc system pl}
\item \textbf{a10soc system}
\end{itemize}
The next sections document the module in great detail.

